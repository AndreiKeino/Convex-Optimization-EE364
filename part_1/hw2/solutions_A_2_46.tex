\documentclass{article}
\usepackage{graphicx}
\usepackage{amsmath}

\begin{document}

\title{Solutions to hw2 homework on Convex Optimization https://web.stanford.edu/class/ee364a/homework.html}
\author{Andrei Keino}
\maketitle
% 3.22, 3.28, 3.39, A2.23, A2.42, A2.46
\section*{2.46}

Curvature of some order statistics. For $x \in R^n$, with $n > 1$, $x_{[k]} $ denotes the kt-h largest entry
of $x$, for $k = 1,...,n$, so, for example, $x_{[1]} = max_{i=1,...,n} \, x_i$ and $x_{[n]} = min_{i=1,...n} \, xi$. Functions
that depend on these sorted values are called order statistics or order functions. Determine the
curvature of the order statistics below, from the choices convex, concave, or neither. For each
function, explain why the function has the curvature you claim. If you say it is neither convex nor
concave, give a counterexample showing it is not convex, and a counterexample showing it is not
concave. All functions below have domain $R^n$. \\

(a) $median(x) = x_{[(n+1)/{2}]}$. (You can assume that n is odd.)
\\ Neither convex nor concave. The average of median of $(2, 0, 0)$ and $(0, 2, 0)$ is $0$, but median of their average $(1, 1, 0)$ is $1$, that violates convexity. The median of $(0, 0, -2)$ and $(0, -2, 0)$ is $0$, but median of their average $(0, - 1, -1)$ is $-1$, that violates concavity.\\

(b) The range of values, $x_{[1]} - x_{[n]}$. \\
This is a convex function because this is a sum of two convex functions.\\

(c) The midpoint of the range, $(x_{[1]} + x_{[n]}) / 2$.\\
Neither convex nor concave. The average of midpoint of $(1, 1, 0)$ and $(0, 1, 1)$ is $1/2$ and midpoint of their average is $3/4$, that violates convexity. The midpoint of $(1, 0, 0)$ and $(0, 0, 1)$ is $1/2$ and midpoint of their average is $1/4$, that violates concavity.\\

(d) Interquartile range, defined as $x_{[n/4]} - x_{[3n/4]}$. (You can assume that n/4 is an integer.) \\
Neither convex nor concave. The interquartile range of 
$(0, 1, 1, 1, 1)$ and $(1, 0, 1, 1, 1)$ is $0$ and interquartile range of their average is $1/2$, that violates convexity. The interquartile range of $(0, 0, 0, -1, -1)$ and $(0, 0, 0, 1, 1)$ is $1$ and interquartile range of their average is $0$, that violates concavity.

(e) Symmetric trimmed mean (STM), defined as \\
$$\frac{x_{[n / 10]} + x_{[n / 10 + 1]} + ... + x_{[9n / 10]}}
	{0.8 n + 1}$$\\
	
Neither convex nor concave. The STM of $x \in R^{20}: x_1 = 1, 
x_2 = 1 x_i = 0 \, \forall \, i \in (3,..., 20)$ is $0$ and the STM of $y \in R^{20}: y_3 = 1, y_4 = 0, y_i = 1 \, \forall \, i \neq 
(3, 4)$ is $0$, but the STM of their average is $1/17$, that violates convexity.  The STM of $u = \boldsymbol 1 - x$ is $10/17$ and the STM of $v = \boldsymbol 1 - y$ is $10/17$, but the STM of their average is $19/34$, that violates concavity. \\

(f) Lower trimmed mean, defined as 
$$\frac{x_{[1]} + x_{[2]} + ... + x_{[9n/10]}}{0.9 n + 1} $$ \\

It's convex. This is the sum of the $9n/10$ largest elements, which is convex, multiplied by a constant.

\end{document}

