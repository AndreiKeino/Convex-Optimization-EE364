\documentclass{article}
\usepackage{graphicx}
\usepackage{amsmath}
\usepackage{hyperref}
\usepackage{float}
\usepackage{xcolor}
\usepackage{enumitem}

\begin{document}

\title{Solutions to hw1 homework on Convex Optimization https://web.stanford.edu/class/ee364b/homework.html}
\author{Andrei Keino}
\maketitle

\section*{1.1 (3 points)} 

For each of the following convex functions, determine the subdifferential set at the specified point.\\

(a) $f(x_1, x_2, x_3) = max(|x_1|, |x_2|, |x_3|)$ at 
$(x_1, x_2, x_3) = (0, 0, 0).$ \\

(b) $f(x) = e^{|x|}$ ($x$ is scalar)\\

(c) $f(x_1, x_2) = max(x_1 + x_2 - 1, x_1 - x_2 + 1)$
at $(x_1, x_2) = (1, 1).$

Solution:

(a) There will be a gap in differential at the points 
$\{x_1 = \pm x_2, x_2 = \pm x_3, x_1 = \pm x_3\}.$
Subdifferential set $g(0, 0, 0) = \{[-1, 1], [-1, 1], [-1, 1]\}.$ \\

(b) There will be a gap in differential at the point $x = 0.$ Subdifferential set $g(0) = [-e^0, e^0] = [-1, 1].$ \\

(c) There will be a gap in differential at the points 
$\{x_1 + x_2 - 1 = x_1 - x_2 + 1\}.$
Subdifferential set $g(1, 1) = \{1, [-1, 1]\}.$

\section*{1.3 (2 points)}
Convex functions that are not subdifferentiable. Verify that the following
functions, defined on the interval $[0;1),$ are convex, but not subdifferentiable at $x = 0.$
(Hint: You can prove by contradiction.)\\ 

(a) $f(0) = 1$ and $f(x) = 0$ for $x > 0.$ \\

(b) $f(x) = -x^p$ for some $p \in (0, 1)$\\

Solution. \\

(a) Proof by contradiction. 
Suppose what function 
$f(0) = 1$ and $f(x) = 0$ for $x > 0.$ has a supporting hyperplane at point $x = 0,$ and g is the subgradient of $f(x)$ in this point. Then at $x \geq 0$ the equation
$f(x) \geq f(0) + g x$ must hold. For $x > 0$ this equation become $0 \geq 1 + g x $ or $gx \leq - 1$ for $x \geq 0.$ This is impossible, because at $x = 0$ we must have $0 \leq -1 $ then. \\

(b) Proof by contradiction.
Suppose what function $f(x) = -x^p$ for some $p \in (0, 1)$ has a supporting hyperplane at point $x = 0,$ and g is the subgradient of $f(x)$ in this point. Then $\forall$ $x \geq 0$ the equation $f(x) \geq f(0) + g x$ must hold. But this is impossible, as f(0) is $\infty$ (i.e.  unlimited) and $f(x)$ has a limited value, i.e $g$ should be unlimited in this case.\\

\section*{1.2 (7 points)}
 For each of the following convex functions, explain how to calculate a subgradient at a given x.\\
 
 (a) $f(x) = max_{i = 1, \dots, m}(a_i^T x + b_i).$\\
 
 (b) $f(x) = max_{i = 1, \dots, m}(|a_i^T x + b_i|).$\\
 
 (c) $f(x) = max_{i = 1, \dots, m}(- 
 log(a_i^T x + b_i).$ . You may assume x is in the domain of $f.$\\
 
 (d) $f(x) = max_{0 \leq t \leq 1}(p(t))$ where 
 $p(t) = x_1 + x_2 t + \dots + x_n t^{n - 1}.$\\
 
 (e) $f(x) = x_{[1]} + \dots + x_{[k]}$ where $x_{[i]}$ 
 denotes the $i-$ th largest element of $x.$ \\
 
 (f) $f(x) = min_{Ay \preceq b}(||x^2 - y^2||),$ 
 , i.e., the square of the distance of $x$ to the polyhedron defined by $Ay \preceq b.$ You may assume that the inequalities $Ay \preceq b.$ are strictly feasible. (Hint: You may use duality, and then use subgradient the rule for pointwise maximumum.  \\
 
 (g) $f(x) = max_{Ay \preceq b}(y^t x),$  x, i.e., the optimal value of an LP as a function of the cost
 vector. (You can assume that the polyhedron defined 
 $Ay \preceq b$ is bounded.)
 (Hint: You may use the subgradient rule for pointwise maximum.\\
 
 Solution. \\
 
 (a) Find $k \in {1, \dots, m}$ such that 
 $f(x) = a_k^T x + b_k.$ Then subgradient at this point is $a_k.$
 
 (b) Find $k \in {1, \dots, m}$ such that 
 $f(x) = |a_k^T x + b_k|.$ If $a_k^T x + b_k > 0$ then subgradient 
 is $a_k,$ if $a_k^T x + b_k < 0$ then subgradient 
 is $- a_k,$ if $a_k^T x + b_k = 0$ then subgradient 
 is $[-|a_k|, |a_k|].$ \\
 
 (c) Find $k \in {1, \dots, m}$ such that 
 $f(x) = -log(a_k^T x + b_k).$ Then subgradient is 
 $- 1 / (a_k^T x + b_k).$\\
 

 (d) Find $t$ such that 
$f(x) = x_1 + x_2 t + \dots + x_n t^{n - 1}.$ Then subgradient is $(1, t, \dots , t^{n-1}).$ \\

 (e) Find $\{i_1, \dots, i_k\}$ such that 
 $f(x) = x_{i_1} + \dots + x_{i_k}.$ Then subgradient is
 $(a_1, \dots, a_n)$ where $a_j = 1$ if 
 $j \in \{i_1, \dots, i_k\},$ $a_j = 0$ otherwise. \\
 
  (f) $f(x)$ defined as optimal value of the problem
  
\begin{align*}
	&\text{minimize } && ||x - y||^2 \\
	&\text{subject to } && Ay \preceq b\\
\end{align*}
With variable y.
We have dual problem because Slater’s condition holds. 
\begin{align*}
	&\text{minimize } && 
	-1/4 z^TAA^Tz + 1/2 z^T A x - b^Tz \\
	&\text{subject to }  && z \succeq 0\\
\end{align*}

From the global perturbation inequalities 
$$
f(x, b) \geq f(x^*, b) - z^{*T}(x - x^*)
$$
By Slater’s condition, we have strong duality and the dual optimum is attained.

Let $z*$ be the optimal dual solution for for the value of $x$ at which we want a subgradient, i.e., $z \succeq 0$ and 
$$
f(x) = -1/4 z^{*T}AA^Tz + 1/2 z^{*T} A x - b^Tz^*
$$
By weak duality we have for any $\hat{x},$

$$
f(\hat x) \geq -1/4 z^{*T}AA^Tz + 1/2 z^{*T} A \hat x - b^Tz^* = f(x) + A^T z^* (\hat x - x)
$$

The KKT conditions for $y*$ to be the optimal point of the primal problem gives:
$$
A^T z* = 2(x - y*)
$$
Therefore $(x - y^*)$ is a subgradient at $x.$\\

(g) The set $\{y \;| \; Ay \preceq b \} $ is closed and bounded, i.e. compact.This means that
the supremum in the definition of $f(x)$ is attained.
Let $\hat y \in \{y\; A y \succeq b\}$ be the value of $y$ for which $f(x) = \hat y^t x.$ Then $\hat y$ is a subgradient of $f$ at x.





  
 
 


 
 
 

\end{document}