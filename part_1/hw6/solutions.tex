\documentclass{article}
\usepackage{graphicx}
\usepackage{amsmath}
\usepackage{hyperref}
\usepackage{float}
\usepackage{xcolor}

\begin{document}

\title{Solutions to hw6 homework on Convex Optimization https://web.stanford.edu/class/ee364a/homework.html}
\author{Andrei Keino}
\maketitle

% Homework 6, due Friday 8/7/20:
% 7.7, 8.15, A13.2, A16.3, A18.2, A18.9.

% 7.7 p. 395
\section*{7.7}
ML estimation of Poisson distributions. Suppose $x_i,$ $i = 1, \dots , n$ are independent random variables with Poisson distributions 
$$
\boldsymbol{prob}(x_i = k) = \frac{e^{-\mu_i} \mu_i^k}{k!}
$$
with unknown means $\mu_i.$ The variables $x_i$ represent number of times that one of $n$ possible independent events occurs during a certain period. In emission tomography, for
example, they might represent the number of photons emitted by $n$ sources. We consider an experiment designed to determine the means $\mu_i.$ The experiment involves $m$ detectors. If event i occurs, it is detected by detector $j$ with probability $p_{ij}.$ We assume
the probabilities $p_{ij}$ are given (with $p{ij} \geq 0,$ 
$\sum_{j =1}^{m}p_{ij} \leq 1$). The total number of events
recorded by detector $j$ is denoted $y_j$,
$$
y_j = \sum_{i = 1}^{m} y_{ij}, \quad i = 1, \dots , m
$$

Formulate the ML estimation problem of estimating the means 
$\mu_i$ based on observed values of $y_j,$ $j = 1, \dots ,m,$ as
 a convex optimization problem.
 
 Hint: the variables $y_{ij}$ have the Poisson distribution with means $p_{ij} \mu_i, $ i.e 
$$
\boldsymbol{prob}(y_{ij} = k) = 
\frac{e^{-p_{ij}\mu_i} (p_{ij} \mu_i)^k}{k!}
$$
The sum of $n$ independent random variables with means 
$\lambda_1, \dots \lambda_n$ has a Poisson distribution with mean $\lambda_1 +  \dots + \lambda_n.$\\

Solution:

It follows from the hint, that 
$$
\boldsymbol{prob}(y_j = k) = 
\frac{e^{-p_j^T \mu} (p_j^T\mu)^k}{k!}
$$
Then likelihood of the set of $y_j = k_j,$ $j = 1, \dots , m$ is 

$$
L\{k_j\} = 
\prod_{j = 1}^{m}\frac{e^{-p_j^T \mu} (p_j^T\mu)^{k_j}}{k_j!}
$$
and the log likelihood is

$$
LL\{k_j\} = 
\sum_{j = 1}^{m}(-p_j^T \mu + k_j log(p_j^T\mu) - log(k_j!))
$$
$LL\{k_j\}$ is a concave function of $\mu_j.$ So, 
 the solution is:

\begin{align*}
\text{maximize } &
\sum_{j = 1}^{m}(-p_j^T \mu + k_j log(p_j^T\mu)) \\
\text{subject to } & \mu \succeq 0 \\
\end{align*} 

% 8.15 - p. 449
\section*{8.15}
Minimum volume ellipsoid covering union of ellipsoids.\\ Formulate the following problem
as a convex optimization problem. Find the minimum volume ellipsoid $\mathcal{E} = \{x \;| \; 
(x - x_0)^T A^{-1} (x - x_0) \leq 1\}$ that contains $K$ given ellipsoids 
$$
\mathcal{E}_i = \{x \;| \; (x)^T A_i x + 2b_i^Tx + c_i \leq 1\}
$$
Hint. See appendix \emph{{\color{red}B}}.\\
% appendix B - p. 653

Solution:\\ \\
$\mathcal{E}$ contains $\mathcal{E}_i$ if \\
$\forall x: \, x^T A^{-1} x - 2 x_0^T A^{-1} x + x_0^T A^{-1}x_0 - 1 \leq 0$ \\
$(x)^T A_i (x) + 2b_i^Tx + c_i - 1 \leq 0$\\
From the S - procedure in the appendix \emph{{\color{red}B}} we have it's true if there exists no such $\lambda \geq 0$ that:\\

$$
\lambda_i
\begin{bmatrix}
A_i& b_i\\
b_i^T & c_i
\end{bmatrix} -
\begin{bmatrix}
A^{-1} & -A^{-1}x_0\\
-(A^{-1}x_0)^T & x_0A^{-1}x_0 - 1\\
\end{bmatrix}
  \succeq 0
$$
this is the same as:

$$
\begin{bmatrix}
	\lambda_i A_i& \lambda_i b_i\\
	\lambda_i b_i^T & \lambda_i c_i + 1
\end{bmatrix} - 
\begin{bmatrix}
I\\
-x_0^T
\end{bmatrix} A^{-1} 
\begin{bmatrix}
I &
-x_0^T
\end{bmatrix}
\succeq 0
$$

So, we have SDP:

\begin{align*}
	\text{minimize } &log(det(A^{-1})) \\
	\text{subject to } &
	\begin{bmatrix}
		\lambda_i A_i&& \lambda_i b_i\\
		\lambda_i b_i^T && \lambda_i c_i + 1
	\end{bmatrix} - 
	\begin{bmatrix}
		I\\
		-x_0^T
	\end{bmatrix} A^{-1} 
	\begin{bmatrix}
		I &&
		-x_0^T
	\end{bmatrix}
	\succeq 0, \; i = 1, \dots, K\\
	& \lambda_i \geq 0, \; i = 1, \dots, K	
\end{align*} 

\section*{A13.2} % p. 148

Optimal spacecraft landing. We consider the problem of optimizing the thrust problem for a spacecraft
to carry out a landing at a target position. The spacecraft dynamics are
$$
m\ddot{p} = f - mge_3
$$
where $m > 0$ is the spacecraft mass, $p(t) \in R^3$ is the spacecraft position with 0 the target landing position and $p_3(t)$ representing height, 
$f(t) \in R^3$ is the thrust force and $g$ is the gravitational acceleration. (For simplicity we assume that the spacecraft mass is constant. This is not always
a good assumption, since the mass decreases with fuel use. We will also ignore any atmospheric
friction.) We must have $p(T^{td}) = 0$ and 
$\dot{p}(T^{td}) = 0$ where $T^{dt}$ is the touchdown time. The spacecraft must remain in a region given by
$$
p_3(t) \geq \alpha ||p_1(t), p_2(t)||_2
$$

where $\alpha > 0$ is a minimal given glide slope. The initial position $p(0)$ and velocity $\dot{p}(0)$ are given. The thrust force $f(t)$ is obtained from a single rocket engine on the spacecraft, with a given
maximum thrust; an attitude control system rotates the spacecraft to achieve any desired direction
of thrust. The thrust force is therefore characterized by the constraint $||f(t)||_2 \leq F^{max}.$ fuel
use rate is proportional to the thrust force magnitude, so the total fuel use is
$$
\int_{0}^{T^{td}} \gamma ||f(t)||_2 dt,
$$
where $\gamma > 0$ is the fuel consumption coefficient. The thrust force is discretized in time, i.e., it is
constant over consecutive time periods of length $h > 0$
with $f(t) = f_k$ for $t \in [(k - 1)h, kh]$ for 
$k = 1, \dots, K,$ where $T^{td} = Kh.$ Therefore we have:
\begin{align*}
v_{k + 1} &= v_{k} + (h / m) f_k- hge_3, \\
p_{k + 1} &= p_k + h / 2 (v_{k + 1} + v_k)
\end{align*} 
For simplicity, we will impose the glide slope constraint only at the times $t = 0, h, 2h, \dots, Kh.$\\

(a) Minimum fuel descent. Explain how to find the thrust profile $f_1, \dots, f_k$ that minimizes fuel
consumption, given the touchdown time $T^{td} = Kh$ and discretization time $h.$ \\

(b) Minimum time descent. Explain how to find the thrust profile that minimizes the touchdown time, i.e., $K,$ with $h$ fixed and given. Your method can involve solving several convex optimization problems. \\

(c) Carry out the methods described in parts (a) and (b) above on the problem instance with
data given in \verb!spacecraft_landing_data.*.!
Report the optimal total fuel consumption for
part (a), and the minimum touchdown time for part (b). The data files also contain plotting
code (commented out) to help you visualize your solution. Use the code to plot the spacecraft
trajectory and thrust profiles you obtained for parts (a) and (b).\\

Solution:

(a) To find the minimum fuel consumption profile 
we should minimize the expression 
$\sum_{k = 1}^K ||f_k||_2$. Then the problem is:

\begin{align*}
&\text{minimize} && \sum_{k = 1}^K ||f_k||_2\\ 
&\text{subject to} 
&&v_{k + 1} = v_{k} + (h / m) f_k- hge_3, \\
& &&p_{k + 1} = p_k + h / 2 (v_{k + 1} + v_k)\\
& && p_1 = p(0) \\
& && v_1 = \dot{p}(0)\\
& &&p_{K + 1} = 0 \\
& && v_{K + 1} = 0 \\
& && ||f_k||_2 \leq F^{max}, \; 
p_3(t) \geq \alpha ||p_1(t), p_2(t)||_2, 
\; k = 1, \dots, K\\
\end{align*} 
where $p_i, v_i, f_i, \; i = 1, \dots, K$ are variables.\\

(b) We can solve a sequence of convex feasibility problems tofind the minimum touchdown time. For each $K$ we solve the problem 

\begin{align*}
	&\text{minimize} && 0\\
	&\text{subject to} 
	&&v_{k + 1} = v_{k} + (h / m) f_k- hge_3, \\
	& &&p_{k + 1} = p_k + h / 2 (v_{k + 1} + v_k)\\
	& && p_1 = p(0) \\
	& && v_1 = \dot{p}(0)\\
	& &&p_{K + 1} = 0 \\
	& && v_{K + 1} = 0 \\
	& && ||f_k||_2 \leq F^{max}, \; 
	p_3(t) \geq \alpha ||p_1(t), p_2(t)||_2, 
	\; k = 1, \dots, K\\
\end{align*} 
where $p_i, v_i, f_i, \; i = 1, \dots, K$ are variables.\\
If the problem is feasible, we reduce $K$, otherwise we increase $K$. We iterate until we nd the smallest $K$ for which a feasible trajectory can be found.

(c) The matlab code:

\begin{verbatim}
spacecraft_landing_data;
% solve part (a) (find minimum fuel trajectory)
cvx_solver sdpt3;
cvx_begin
variables p(3,K+1) v(3,K+1) f(3,K)
v(:,2:K+1) == v(:,1:K)+(h/m)*f-h*g*repmat([0;0;1],1,K);
p(:,2:K+1) == p(:,1:K)+(h/2)*(v(:,1:K)+v(:,2:K+1));
p(:,1) == p0; v(:,1) == v0;
p(:,K+1) == 0; v(:,K+1) == 0;
p(3,:) >= alpha*norms(p(1:2,:));
norms(f) <= Fmax;
minimize(sum(norms(f)))
cvx_end
min_fuel = cvx_optval*gamma*h;
p_minf = p; v_minf = v; f_minf = f;
% solve part (b) (find minimum K)
% we will use a linear search, but bisection is faster
Ki = K;
while(1)
cvx_begin
variables p(3,Ki+1) v(3,Ki+1) f(3,Ki)
v(:,2:Ki+1) == v(:,1:Ki)+(h/m)*f-h*g*repmat([0;0;1],1,Ki);

p(:,2:Ki+1) == p(:,1:Ki)+(h/2)*(v(:,1:Ki)+v(:,2:Ki+1));
p(:,1) == p0; v(:,1) == v0;
p(:,Ki+1) == 0; v(:,Ki+1) == 0;
p(3,:) >= alpha*norms(p(1:2,:));
norms(f) <= Fmax;
minimize(sum(norms(f)))
cvx_end
if(strcmp(cvx_status,'Infeasible') == 1)
Kmin = Ki+1;
break;
end
Ki = Ki-1;
p_mink = p; v_mink = v; f_mink = f;
end
% plot the glide cone
x = linspace(-40,55,30); y = linspace(0,55,30);
[X,Y] = meshgrid(x,y);
Z = alpha*sqrt(X.^2+Y.^2);
figure; colormap autumn; surf(X,Y,Z);
axis([-40,55,0,55,0,105]);
grid on; hold on;
% plot minimum fuel trajectory for part (a)
plot3(p_minf(1,:),p_minf(2,:),p_minf(3,:),'b','linewidth',1.5);
quiver3(p_minf(1,1:K),p_minf(2,1:K),p_minf(3,1:K),...
f_minf(1,:),f_minf(2,:),f_minf(3,:),0.3,'k','linewidth',1.5);
print('-depsc','spacecraft_landing_a.eps');
% plot minimum time trajectory for part (b)
figure; colormap autumn; surf(X,Y,Z);
axis([-40,55,0,55,0,105]); grid on; hold on;
plot3(p_mink(1,:),p_mink(2,:),p_mink(3,:),'b','linewidth',1.5);
quiver3(p_mink(1,1:Kmin),p_mink(2,1:Kmin),p_mink(3,1:Kmin),...
f_mink(1,:),f_mink(2,:),f_mink(3,:),0.3,'k','linewidth',1.5);
print('-depsc','spacecraft_landing_b.eps');	
\end{verbatim}	

The optimal total fuel consumption is 193:0. The minimum touchdown
time is K = 25.


\section*{A16.3}


% forms of linear matrix inequality
% LMI tutorial:
% https://link.springer.com/content/pdf/bbm%3A978-1-84628-493-9%2F1.pdf
\end{document}

