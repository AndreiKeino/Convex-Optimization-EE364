\documentclass{article}
\usepackage{graphicx}
\usepackage{amsmath}
\usepackage{hyperref}
\usepackage{float}
\usepackage{xcolor}
\usepackage{enumitem}

\begin{document}

\title{Solutions to hw2 homework on Convex Optimization https://web.stanford.edu/class/ee364b/homework.html}
\author{Andrei Keino}
\maketitle

\section*{2.1 (8 points, 1 point per question)} 
Let $f$ be a convex function with domain in $R^n.$ 
We fix $x \in \bf{int \; dom}\; f  $ and $d \in R^n.$
Recall the definition of the directional derivative of $f$
at $x$ along the direction $d$
$$
f'(x, d) = \lim_{t \rightarrow 0} \frac{f(x + td) - f(x)}{t}
$$
In this question we aim to show that $f'(x, d)$ exists and is finite, and that we have the following relationship 
between $\partial f(x)$  and $f'(x, d),$
$$
f'(x, d) = \sup_{g \in \partial f(x)} g^T d
$$
\\

(a) Show that the ratio $\frac{f(x + td) - f(x)}{t}$ is a nondecrasing function of $t > 0.$ Deduce that $f'(x, d)$
exists and is either finite or equal to $- \infty.$ \\
We know from the lectures that, since $x \in \bf{int \; dom \;} f,$ the subdifferential set $\partial f$ is non - empty, convex and compact.
\\

Solution:\\

\textbf{Proof of non - decreasing.}
Definition of subgradient is
$$
f(z) \geq f(x) + g^T(z - x)
$$
let $z = x + td;$ then
$$
f(x + td) \geq f(x) + g^T(x + td - x)
$$
or 
$$
f(x + td) - f(x) \geq t g^T d
$$
dividing both part of the inequality by $t$ (as $t > 0,$
we can do it) gives

$$
\frac{f(x + td) - f(x)}{t} \geq g^T d
$$
as the right - hand side of the equation is not depends of $t,$ differentiating by $t$  gives
$$
\partial \frac {\frac{f(x + td) - f(x)}{t}} {\partial t} \geq 0
$$


\textbf{As the $\frac{\partial f'(x, d}{\partial t} \geq 0,$ it means the function $f'(x, d)$ is nondecreasing by variable $t.$ }

\textbf{Proof of possible equality to $- \infty.$}\\
The definition of convexity: \\

$$
f(\theta x + (1 - \theta) y)) \leq 
\theta f(x) + (1 - \theta) f(y)
$$
where $0 < \theta < 1.$ \\
let $t = 1 - \theta,$ $0 < t < 1.$
then
$$
f((1 - t) x + t y)) \leq 
(1 - t)f(x) + tf(y)
$$
or
$$
f(x + t (y - x))) \leq 
f(x) + t(f(y) - f(x))
$$
as we can choose $y$ any of the point in domain $f,$ we can set $d = y - x.$ Then
$$
f(x + t d) \leq 
f(x) + t(f(y) - f(x))
$$
or
$$
f(x + t d) - f(x)\leq 
t(f(y) - f(x))
$$
or
$$
\frac{f(x + t d) - f(x)}{t}\leq 
f(y) - f(x)
$$
\textbf{As $f(x)$ can be equal to $\infty$ on the domain of $f$, so
$f'(x, d) = \frac{f(x + t d) - f(x)}{t}$ can be less or equal than (for the infinity with sign minus it means strictly equal) $- \infty$ on the domain of $f.$} This means that $f'(x, d)$ can be equal to $- \infty$ on domain of $f.$\\

(b) Let $g \in \partial f(x).$ Show that 
$f'(x, d) \geq g^T d.$ Deduce that $f'(x, d)$ is finite and $f'(x, d) \geq \sup_{g \in \partial f(x)} g^T d.$ \\

Solution:\\
We already shown that 
$$f'(x, d) \geq g^T d$$
in part (a).
We also shown in part (a) that 
$$
\frac{f(x + t d) - f(x)}{t}\leq 
f(y) - f(x)
$$
Second upper inequality means that  $f'(x, d)$ is bounded from upper side (i.e it can't be equal to $\infty$), it means  its value is finite. \\
As the first of upper inequalities is correct $\forall$ subgradients in domain of $f,$ it means, that it is correct for  the supremum of these subgradients in domain $f.$ It means that 
$$f'(x, d) \geq \sup_{g \in \partial f(x)} g^T d.$$
\\

In the remaining part of this question, we will establish the converse inequality $f'(x, d) \leq \sup_{g \in \partial f(x)} g^T d,$ by showing the existence of a subgradient $g^* \in \partial f(x), $ such that 
$f'(x, d) \leq g^{*T} d.$ We introduce two following sets
\begin{align*}
	C_1 &= \{(z,t) \; |\; z \in \mathbf{dom} f, \; f(z) < t \} \\
	C_2 &= \{(y, v) \; | \; y = x + \alpha d, \;
	v = f(x) + \alpha f'(x, d), \; \alpha \geq 0 \}
\end{align*}

(c) Prove that $C_2$ and $C_2$ are nonempty, convex and disjoint.\\

Solution: \\

$C_1$ epigraph of the convex function, therefore 
it is nonempty and convex.

$C_2$ is the nonempty set, because it have at least one point, which corresponds to $\alpha = 0,$ $y = x,$ $v = f(x).$ It is also a convex set, because 
$C^1_2 = \{y\;| \;y = x + \alpha d\}$ is a convex set as it is translated domain of $f$ which is a convex set
and $C^2_2 = \{v\;| \; 
v = f(x) + \alpha f'(x, d), \; \alpha \geq 0 \}$ is either 
a straight line or a beam or a segment. \\

Proof of disjointedness: \\
We should show that there is exists a nonzero vector 
$(a, \beta) \in R^n \times R$ such as
$$
a^T(x + \alpha d) + \beta (f(x) + \alpha f'(x, d)) \leq
a^T z + \beta w
$$
for all $\alpha \ geq 0, $ $z \in \mathbf{dom} f,$ 
and $f(z) < w.$
\\
Solution:
As we shown earlier, 
$$
f'(x, d) \leq f(y) - f(x)
$$
where $x, y \in \mathbf{dom} f.$ As $x, y$ can be any points in domain $f,$ it follows that\\
$$
f'(x, d) \leq \min_{z \in \mathbf{dom} f }(f(z)) - 
\max_{z \in \mathbf{dom} f}(f(z))
$$
Lets just derive equation for $\beta.$

$$
\beta (f(x) + \alpha f'(x, d) - w) \leq 
a^T(z - x - \alpha d)
$$
or
$$
\beta \leq \frac{a^T (z - x - \alpha d)}
{f(x) + \alpha f'(x, d) - w}
$$

I don't know how to solve items (e) - (g) \\\

(h) Let $A \in R^{m\times n},$ $b \in R^m,$ $\lambda > 0,$ 
and fix a direction $d \in R^n.$ Consider the function 
$\frac{1}{2} ||Ax - b||^2_2 + \lambda ||x||_1.$ Compute 
$f'(0, d).$ Remark: you can either compute it
directly by using the definition of the directional derivative, or, use the variational
formula $f'(0, d) = \sup_{g \in \partial f(0)} g^T d.$
\\

Solution: \\ \\
$
\nabla ||x||_1 = sign(x)
$\\
$\nabla ||Ax - b||^2_2 = (\nabla (Ax - b)^T) (Ax - b) + 
(Ax - b)^T \nabla (Ax - b) = 2(A^TAx - 2 A^T b)
$\\ \\
So, \\
$
\nabla (\frac{1}{2} ||Ax - b||^2_2 + \lambda ||x||_1) = 
A^TAx - A^T b + \lambda sign(x)
$
\\ \\
Then \\
$$
f'(0, d) = d^T (- A^T b + \lambda [-1, 1]_n) 
$$

where $ [-1, 1]_n$ is a vector in $R^n$ with component values in range \\
$-1 \leq x_i \leq 1, \; i \in {1, \dots , n}.$
\end{document}